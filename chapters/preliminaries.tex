\chapter{Preliminaries}\label{chapter:preliminaries}

\section{Space Plasma}\label{sec:plasma}

Plasma is ubiquitous throughout the visible Universe, also known as the fourth state of matter 
due to its properties which differentiate it from the conventional gaseous state. Plasma is a 
gas which is composed of roughly equal number of positive and negatively charged particles, this
property is known as charge \emph{quasi-neutrality}. The term quasi-neutral is used because although
the gas has almost equal amount of positive and negative charges, the mixture is electricomagnetically 
active. Due to incomplete charge sheilding, long range electromagnetic fields play a big role in the 
dynamics of plasma.

From classical electrostatics the electric potential of a point charge $q$, is given as

\begin{equation}
    \phi(r) = \frac{q}{4\pi\epsilon_0 ||r||_2}
\end{equation}

where $r$ is the position in space with respect to the charge and $\epsilon_0$ is the \emph{permitivity} of vacuum.

\subsection*{Debye Length}

In a quasi-neutral plasma, due to the prescence of partial electric sheilding the potential due to the charge 
now takes the so called Debye form.

\begin{equation}
    \phi(r) = \frac{q}{4\pi\epsilon_0 ||r||_2} e^{-\frac{||r||_2}{\lambda_d}}
\end{equation}

The electric potential decays with the Debye length scale $\lambda_d$ at which a balance between thermal vibrations 
which can disturb quasi-neutrality and electrostatic forces due to charge separation. The Debye length scale depends
on the electron temperature and plasma density.

\begin{equation}\label{eq:debye}
    \lambda_d = \sqrt{\frac{\epsilon_0 k_b T_e}{n_e e^2}}
\end{equation}

In equation \ref{eq:debye}, the Debye length scale is expressed in terms of the \emph{Boltzmann constant} $k_b$, 
the electron temperature $T_e$, free space permitivity $\epsilon_0$ and electron charge $e$. One can visualise the 
positively charged ions having a cloud of electrons sheilding them at the distance of $\lambda_d$. 

It is also possible to take into account the sheilding effect of the ions. The effective Debye length is now 
expressed as an addition of two terms, one for electrons (\ref{eq:debye}) and a similar term for the ions by replacing 
$T_e$ for the ion temperature $T_i$ ($n_i \approx n_e$). 

\subsection*{Plasma Parameter}

Consider a Debye sphere of radius $\lambda_d$, this sphere contains $N_e = \frac{4}{3}\pi \lambda^3_d n_e$ electrons. The plasma parameter $g$ is defined as $N_{e}^{-1}$. Rewriting this, we can say:

\begin{equation}
    g \sim \sqrt{\frac{n_e}{T_e}}
\end{equation}

The description of plasma used in many applications in Space is applicable when $g \|| 1$, in this situation 
the Debye sheilding is significant and the quasi-neutral plasma obeys collective statistical behaviour. 
The plasma parameter $g$ also correlates with the collision frequency. As the collisions in plasma increase 
with increasing density and decreasing temperature, if $g \longrightarrow 0$ the plasma becomes nearly collisionless. The collisionless property helps in making simplfying assumtions about plasma dynamics and is the starting point 
for the \emph{adiabatic} theory of plasma motions in the Earth's magnetosphere which will be discussed in section 
\ref{sec:plasmadiff}.


\section{Sun \& the Solar Wind}\label{sec:solar}

\begin{figure}
    \noindent\includegraphics[width=\textwidth]{Sun_poster.png}
    \caption{\small Cross section of the Sun \\ 
    Author: Kelvinsong [CC BY-SA 3.0 (\url{https://creativecommons.org/licenses/by-sa/3.0})] \\ 
    Source: Wikipedia}
    \label{fig:SunLayers}
\end{figure}

The Sun is an almost perfectly spherical ball of plasma which is the dominant star in our solar system and 
the pricipal source of light and energy for all living and meteorological processes on Earth. Apart from 
terrestrial weather, the Sun is also the primary driver of space weather which results from the interaction 
between the solar wind and planetary magnetospheres.

\subsection*{Structure}

Figure \ref{fig:SunLayers} shows a cross section of the Sun with various layers. We give a brief description 
of them below.


\subsubsection*{Core}

The core of the Sun is the site for the thermonuclear fusion reactions which produce its energy. It extends 
from the center to about $20-25\%$ of the solar radius \citep{SolarAct}. It has a temperature close to 
$1.57 \times 10^7$ Kelvin and a density of $150 \text{g}/\text{cm}^3$ \citep{SolarCore}. Nuclear fusion in the core 
takes place via the so called \emph{proton-proton chain} (pp).

\subsubsection*{Radiative Zone}

The radiative zone extends from $25\%$ to $70\%$ of the solar radius. The nuclear reactions in the core are 
highly sensitive to temperature and pressure, in fact they are almost shut off at the edge of the core. 
In the radiative zone, energy transfer takes place via photons (radiation) which bounce around nuclei until 
they reach the Convective zone.

\subsubsection*{Convective Zone}

Between $70\%$ of the solar radius to a point close to the solar surface. Density decreases dramatically going 
from the core to the radiative and subsequently the convective zone. Here the solar material behaves more 
like a fluid. Due to the temperature gradient which exists across it, the primary source of transport is 
via convection.

\subsubsection*{Photosphere}

The photosphere is the visible 'surface' of the Sun, since the layers below it are all opaque to visible light.
A layer of about $100$ kilometre thickness, the photosphere is also the region from where sunlight can freely 
escape into space. The photospheric surface has a number of features i.e. sunspots, granules and faculae. 
Sunspots are magnetic regions where the solar material has lower temperature as compared to its surroundings. 
Magnetic field lines are concentrated in sunspot regions and the field strength in sunspots can often be thousands 
of times stronger than the on the Earth.

\subsubsection*{Chromosphere}

Extending for a distance of almost $5000$ kilometers after the photosphere, the chromosphere has a red colour which 
is generally not visible due to the intense light given off by the photosphere. The chromosphere is known for the 
existense of features called \emph{spicules} and prominences. 


\subsubsection*{Solar Transition Region}

\subsubsection*{Corona}

\subsection*{Solar Wind}

\begin{wrapfigure}{l}{0.4\textwidth}
    \centering\includegraphics[width=0.38\textwidth]{parker-spiral.jpg}
    \caption{
        \small Solar Wind. Image reproduced from \citet{Owens2013}}
    \label{fig:spacex}
\end{wrapfigure}


The existence of the solar wind was first postulated by \citet{ParkerSolarWind}.

\subsection*{Sunspots \& Solar Cycle}

\section{Magnetosphere}\label{sec:mag}

\subsection*{Particle Motions \& Adiabatic Theory} \label{sec:plasmadiff}

\clearpage

\bibliographystyle{plainnat}
\bibliography{references}
