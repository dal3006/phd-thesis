\chapter{Preliminaries}\label{chapter:preliminaries}

\section{Space Plasma}\label{sec:plasma}

Plasma is ubiquitous throughout the visible Universe, also known as the fourth state of matter 
due to its properties which differentiate it from the conventional gaseous state. Plasma is a 
gas which is composed of roughly equal number of positive and negatively charged particles, this
property is known as charge \emph{quasi-neutrality}. The term quasi-neutral is used because although
the gas has almost equal amount of positive and negative charges, the mixture is electricomagnetically 
active. Due to incomplete charge sheilding, long range electromagnetic fields play a big role in the 
dynamics of plasma.

From classical electrostatics the electric potential of a point charge $q$, is given as

\begin{equation}
    \phi(r) = \frac{q}{4\pi\epsilon_0 ||r||_2}
\end{equation}

where $r$ is the position in space with respect to the charge and $\epsilon_0$ is the \emph{permitivity} of vacuum.

\subsection*{Debye Length}

In a quasi-neutral plasma, due to the prescence of partial electric sheilding the potential due to the charge 
now takes the so called Debye form.

\begin{equation}
    \phi(r) = \frac{q}{4\pi\epsilon_0 ||r||_2} e^{-\frac{||r||_2}{\lambda_d}}
\end{equation}

The electric potential decays with the Debye length scale $\lambda_d$ at which a balance between thermal vibrations 
which can disturb quasi-neutrality and electrostatic forces due to charge separation. The Debye length scale depends
on the electron temperature and plasma density.

\begin{equation}\label{eq:debye}
    \lambda_d = \sqrt{\frac{\epsilon_0 k_b T_e}{n_e e^2}}
\end{equation}

In equation \ref{eq:debye}, the Debye length scale is expressed in terms of the \emph{Boltzmann constant} $k_b$, 
the electron temperature $T_e$, free space permitivity $\epsilon_0$ and electron charge $e$. One can visualise the 
positively charged ions having a cloud of electrons sheilding them at the distance of $\lambda_d$. 

It is also possible to take into account the sheilding effect of the ions. The effective Debye length is now 
expressed as an addition of two terms, one for electrons (\ref{eq:debye}) and a similar term for the ions by replacing 
$T_e$ for the ion temperature $T_i$ ($n_i \approx n_e$). 

\subsection*{Plasma Parameter}

Consider a Debye sphere of radius $\lambda_d$, this sphere contains $N_e = \frac{4}{3}\pi \lambda^3_d n_e$ electrons. 
The plasma parameter $g$ is defined as $N_{e}^{-1}$. Rewriting this, we can say:

\begin{equation}
    g \sim \sqrt{\frac{n_e}{T_e}}
\end{equation}

The description of plasma used in many applications in Space is applicable when $g \|| 1$, in this situation 
the Debye sheilding is significant and the quasi-neutral plasma obeys collective statistical behaviour. 
The plasma parameter $g$ also correlates with the collision frequency. As the collisions in plasma increase 
with increasing density and decreasing temperature, if $g \longrightarrow 0$ the plasma becomes nearly collisionless. 
The collisionless property helps in making simplfying assumtions about plasma dynamics and is the starting point 
for the \emph{adiabatic} theory of plasma motions in the Earth's magnetosphere which will be discussed in section 
\ref{sec:plasmadiff}.

\begin{figure}
    \noindent\includegraphics[width=\textwidth]{Sun_poster.png}
    \caption{\small Cross section of the Sun \\ 
    Author: Kelvinsong [CC BY-SA 3.0 (\url{https://creativecommons.org/licenses/by-sa/3.0})] \\ 
    Source: Wikipedia}
    \label{fig:SunLayers}
\end{figure}


\section{Sun \& the Solar Wind}\label{sec:solar}

The Sun is an almost perfectly spherical ball of plasma which is the dominant star in our solar system and 
the pricipal source of light and energy for all living and meteorological processes on Earth. Apart from 
terrestrial weather, the Sun is also the primary driver of space weather which results from the interaction 
between the solar wind and planetary magnetospheres.

\subsection{Structure}

Figure \ref{fig:SunLayers} shows a cross section of the Sun with various layers. We give a brief description 
of them below.

\textbf{Core}: The core of the Sun is the site for the thermonuclear fusion reactions which produce its energy. It extends 
from the center to about $20-25\%$ of the solar radius \citep{SolarAct}. It has a temperature close to 
$1.57 \times 10^7$ Kelvin and a density of $150 \text{g}/\text{cm}^3$ \citep{SolarCore}. Nuclear fusion in the core 
takes place via the so called \emph{proton-proton chain} (pp).

\begin{wrapfigure}{r}{0.4\textwidth}
    \centering\includegraphics[width=0.38\textwidth]{chromosphere.jpg}
    \caption{
        \small Chromosphere when viewed using an $H\alpha$ filter. Source: CWitte [Public domain]}
    \label{fig:chromosphere}
\end{wrapfigure}

\textbf{Radiative Zone}: The radiative zone extends from $25\%$ to $70\%$ of the solar radius. The nuclear reactions 
in the core are highly sensitive to temperature and pressure, in fact they are almost shut off at the edge of the core. 
In the radiative zone, energy transfer takes place via photons (radiation) which bounce around nuclei until 
they reach the Convective zone.

\textbf{Convective Zone}: Between $70\%$ of the solar radius to a point close to the solar surface. Density decreases 
dramatically going from the core to the radiative and subsequently the convective zone. Here the solar material behaves more 
like a fluid. Due to the temperature gradient which exists across it, the primary source of transport is 
via convection.

\textbf{Photosphere}: The photosphere is the visible 'surface' of the Sun, since the layers below it are all opaque to 
visible light.A layer of about $100$ kilometre thickness, the photosphere is also the region from where sunlight can freely 
escape into space. The photospheric surface has a number of features i.e. sunspots, granules and faculae. 
Sunspots (see section\ref{sec:sunspots}) are magnetic regions where the solar material has lower temperature as compared 
to its surroundings. Magnetic field lines are concentrated in sunspot regions and the field strength in sunspots can often 
be thousands of times stronger than the on the Earth.

\textbf{Chromosphere}: Extending for a distance of almost $5000$ kilometers after the photosphere. The chromosphere is 
known for the existense of features called \emph{spicules} and prominences. The chromosphere has a red colour which 
is generally not visible due to the intense light given off by the photosphere, but can be observed through 
a filter centered on the Hydrogen $H\alpha$ spectral line (see figure \ref{fig:chromosphere}). 

\textbf{Solar Transition Region}: A thin ($100 \text{km}$) region between the chromosphere and the solar corona 
where the temperature rises from about $8000 \text{K}$ to $500000 \text{K}$. The transition region might not be well 
defined at all altitudes, but its existence is evidenced by a bifurcation of the dynamics of the solar plasma. 
Below the transition region, the dynamics is dictated by gas pressure, fluid dynamics and gravitation while 
above the region the dynamics is dictated more by magnetic forces.

\textbf{Corona}: An aura of plasma around the Sun that extends millions of kilometers into space, the corona can be 
observed during a total solar eclipse or with a coronograph. The temperature of the corona is dramatically higher 
than the photosphere and chromosphere. The average temperature can range between 
$1 \times 10^6 \ \text{to} \ 2 \times 10^6 \text{K}$ while in the hottest regions it can be as high as 
$2 \times 10^7 \text{K}$ \citep{SolarCorona}. Although the reason for this dramatic increase is still 
not well understood, there various explanations using concepts of magnetic reconnection 
\citep{russell2001solar,SolarCorona} and Alfv\'en waves \citep{AlfvenCorona}.

\begin{figure}
    \noindent\includegraphics[width=0.8\textwidth]{parker-spiral.jpg}
    \caption{\small An illustration of the Heliospheric Magnetic Field in the \emph{ecliptic plane}. 
    In the heliosphere, rotation of the HMF footpoints within a radial solar wind flow generates an azimuthal 
    component of the HMF, $B_{\phi}$, leading to a spiral geometry. Red and blue lines, 
    showing regions of opposite polarity, are separated by the heliospheric current sheet (HCS), 
    shown as the green dashed line.
    Image reproduced from \citet{Owens2013}}
    \label{fig:parkerspiral}
\end{figure}

\subsection{Heliospheric Magnetic Field \& Solar Wind}

The existence of the solar wind was first postulated via observations of Halley's coment in 
\citet{Bierman1,Bierman2,Bierman3} and subsequently models of the \emph{Heliospheric Magnetic Field} (HMF) 
were introduced in \citet{parker1958dynamics}. These models were further evidenced by their ability 
to explain the effect of the HMF on the modulation of galactic cosmic rays and their measured intensities 
close to the Earth \citep{ParkerSolarWind}. 

The HMF in steady state points radially outward and rotates with the Sun, producing an \emph{Archimedian spiral} 
structure as postulated in \cite{parker1958dynamics} and shown schematically in figure \ref{fig:parkerspiral}. 
Photospheric observations of the magnetic field (see Global Oscillation Network Group \url{https://gong.nso.edu}) 
are often extrapolated to compute approximations to the HMF. There exist a number of models used to perform such 
extrapolations, such as the \emph{Potential-Field Source Surface} model (PFSS) 
\citep{schatten1969model,altschuler1969magnetic}, the \emph{Current-Sheet Source Surface} model (CSSS) 
\citep{csss} and the \emph{Magnetohydrodynamics Around a Sphere} model (MAS) 
\citep{linker1999magnetohydrodynamic}. 

The HMF can be seen as a combination of two components: the poloidal magnetic field and the toroidal magnetic field.
The two fields often exchange energy between themselves over the course of several years in a cyclical phenomenon
known as the \emph{solar cycle} (section \ref{sec:sunspots}). Interested readers can read \citep{Owens2013} for an 
in-depth review on the phenomena that drive the HMF.

\begin{wrapfigure}{r}{0.4\textwidth}
    \centering\includegraphics[width=0.38\textwidth]{solarwinddist.pdf}
    \caption{
        \small Distribution of solar wind speed recorded at 1 AU for the time period $2008 - 2018$}
    \label{fig:solarwinddist}
\end{wrapfigure}

The expansion of the coronal magnetic field leads to an eventual opening of field lines at the source surface 
(see figure \ref{fig:parkerspiral}) and the ejection of the solar wind. This hot plasma consists mostly of protons, 
electrons and a small number of helium and heavy ions. The solar wind spirals outwards in all directions, carrying 
with it the magnetic field. Upon reaching a distance of $1 \text{AU}$ (close to the Earth's magnetosphere), this wind 
has a nominal speed of about $400 \text{km}/\text{s}$ while its high speed component has an average velocity of 
$\sim 700 \text{km}/\text{s}$.



\subsection{Sunspots \& Solar Cycle}\label{sec:sunspots}

Sunspots are temporarily occuring regions on the Sun's photosphere which appear as dark spots. They are areas 
of magnetic field concentration where the field lines often 'puncture' the solar surface inhibiting convection 
and producing regions with lower temperature than the surroundings. Sunspots generally last anywhere between a 
few days to a few months, they can occur in pairs or groups and can accompany other phenomena such as 
\emph{coronal loops}, \emph{prominences} and reconnection events.

Since the $19^{\text{th}}$ century the number of sunspots on the Sun's surface have been recorded as the 
\emph{sunspot number} (SSN). Sunspots populations increase and decrease behaving as markers for solar activity levels, 
this cyclical behaviour is called the \emph{sunspot cycle} or \emph{solar cycle} (figure \ref{fig:SolarCycle}). 

\begin{figure}
    \noindent\includegraphics[width=\textwidth]{sunspot-cycle.jpeg}
    \caption{\small The sunspot butterfly diagram.  \\ 
    Author: David Hathaway, NASA, Marshall Space Flight Center [Public domain] \\ 
    Source: Wikipedia}
    \label{fig:SolarCycle}
\end{figure}

In figure \ref{fig:SolarCycle} we can see how the area occupied by sunspots changes with solar latitude and time. 
During the start of a solar cycle (solar minimum) sunspots start appearing at higher latitudes, over the course 
of the cycle they move towards the equatorial regions and their number increases to some maximum (solar maximum), 
towards the end the number of sunspots diminishes and the entire cycle starts over. This repititive behaviour 
happens over approximately $11$ years.

Because sunspots are magnetic phenomena, the solar cycle represents cyclical behaviour of the HMF. During solar 
minimum the poloidal component of the solar magnetic field is at its strongest and it is closest it can get to 
a magnetic dipole configuration. Towards solar maximum energy is transferred from the poloidal component to the 
toroidal component resulting in complex field configurations which are evidenced by larger numbers of sunspot 
clusters.

The solar cycle also gives rise to variations in solar irradiance \citep{solarirradiance}. Between $1645$ and 
$1715$, a period known as the \emph{Maunder minimum} very few sunspots were observed. This coincided with lower 
than average temperatures in Europe, which was called the \emph{little ice age}. Sunspot numbers going back $11000$ 
years have been reconstructed using Carbon-$14$ based analysis of tree rings. Apart from the \emph{Maunder minimum},
other periods of very low sunspot activity have been matched with corresponding \emph{little ice ages}.

\section{Magnetosphere}\label{sec:mag}

\subsection{Particle Motions \& Adiabatic Theory} \label{sec:plasmadiff}

\clearpage

\bibliographystyle{plainnat}
\bibliography{references}
