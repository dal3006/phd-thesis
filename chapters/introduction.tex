\chapter{Introduction}\label{chapter:introduction}

\epigraph{Weather forecast for tonight: dark.}{\textit{George Carlin}}

\begin{wrapfigure}{l}{0.4\textwidth}
    \centering\includegraphics[width=0.38\textwidth]{MokoShurai.jpg}
    \caption{The Mongol fleet destroyed in a typhoon, ink and water on paper, by Kikuchi Y\={o}sai, 1847. 
    Source: Wikipedia}
    \label{fig:mongolJapan}
\end{wrapfigure}

\emph{Earth}, \emph{Wind}, \emph{Fire} \& \emph{Water}, the \emph{classical elements} were the basis 
for understanding our environment during antiquity. Modern science, based on experiments has taken a very 
different view of the world, one based on atoms, fundamental particles and states of matter. But we could 
argue that the classical elements were a more a philosophical idea that distilled our everyday experiences 
with nature, infact many ancient cultures such as Hellenistic Greece, Babylonia, Japan, Tibet, China and 
India had similar lists of four or five elements. These civilizations had very different views on the 
properties of these elements and how they related to natural phenomena, quite often these links 
were mythological. Indeed the obvious way in which people experienced the classical elements was through
weather systems. 

From the seasons to daily variations, nature's elements drive and shape our lives. Sometimes weather 
has had a direct impact on entire populations, one example was the failed Mongol invasions of Japan 
in $1274$ and $1281$. In both attacks, the Mongol fleets were almost entirely destroyed by storms called 
\emph{kamikaze} (translates to divine wind). Although some attacking Mongol forces did manage to land during 
the $1274$ campaign and outnumbered the defending armies, they were still defeated by Samurai clans with 
superior knowledge of the terrain. 

The invading fleet of $1281$ was composed of "more than four thousand ships bearing nearly $140,000$ men" 
\citep[pg.~17]{mcclain2002japan}, the scale of which was eclipsed only by the allied invasion of Normandy 
in 1944. The fleet was a hastily assembled, consisting of ships which were not suitable for the harsh waters 
between Japan and Korea. The Japanese had built two metre high walls in the intervening period and the invading 
fleets were forced to stay in sea for months. After their supplies were diminished, powerful kamikaze winds 
destroyed them entirely (an artists' view of the event is in figure \ref{fig:mongolJapan}). The failed invasions 
were a blow to the idea of Mongol supremacy in Asia and the Mongols never attemped an invasion of Japan since.

We now know that weather phenomena are caused by a combination of air pressure, temperature and moisture differences 
between one place and another. The angle of the Sun's rays changes with latitude, these variations create very 
different temperature trends from the poles to the equator. These differences in temperature lead to large scale 
air currents which create complex weather systems and climate patterns which we see across the world.

But weather phenomena are hardly exclusive to planet Earth.

\section*{The Final Frontier}

\begin{wrapfigure}{r}{0.4\textwidth}
    \centering\includegraphics[width=0.38\textwidth]{Great_Red_Spot_From_Voyager_1.jpg}
    \caption{
        Jupiter's Great Red Spot in February 1979, photographed by the unmanned Voyager 1 NASA space probe. 
        Source: Wikipedia
    }
    \label{fig:jupiter}
\end{wrapfigure}

Even before the beginning of the space age, weather phenomena occurring on other planets have been observed. Jupiter's 
\emph{great red spot}, a huge storm, has been continuously observed since 1830 \citetext{see \citealp{britannicaRedSpot}}.
Saturn's \emph{great white spot}, a recurring storm system which was first used by Asaph Hall to determine the period of
the planet's rotation \citep{wikisaturn}. In the $20^{\text{th}}$ century, missions such as the Hubble space telescope, 
Voyager, Cassini and others have shown storms and other weather phenomena on planetary bodies like Venus, Mars, 
Neptune and Titan. 

The principles behind many planetary weather phenomena are very similar, their differences are because of the different 
compisition of each planet's atmosphere. Extra-terrestrial weather is just as complex and mind boggling as weather we 
observe on Earth, its scale is certainly much larger than we are used to. 

Yet, planetary weather is just one side of the puzzle. Venturing into our cosmic neighbourhood, our solar system
has another kind of weather system that has begun to be probed only very recently. 

\subsection*{A Gust of Wind from the Heavens}

During the last week of August 1859, several spots appeared on the surface of the Sun. Southern auroral displays were 
observed on August 29, as far north as Queensland Australia. Just before noon on September 1, British astronomer 
Richard Carrington observed a "white light flare" from a group of sun spots. He created a sketch of his observations
which is seen in figure \ref{fig:carringtonevent}. Carrington's observations were independently verified by British 
publisher and astronomer Richard Hodgson, both of them sent their reports to the 
\emph{Monthly Notices of the Royal Astronomical Society}.

September 1-2 1859 saw some remarkable events occur around the world. Auroral displays were observed all around the 
world, even in low latitude places such as Colombia \citep{MORENOCARDENAS2016257}. Auroras above the rocky mountains
in the U.S were so bright that they woke up gold miners who began preparing breakfast thinking it was morning 
\citep{miners}. In the northeastern U.S, people could read the newspaper by the aurora's light \citep{auroraReading}.

\begin{wrapfigure}{l}{0.4\textwidth}
    \centering\includegraphics[width=0.38\textwidth]{Carrington_Richard_sunspots_1859.jpg}
    \caption{Sunspots of September 1, 1859, as sketched by Richard Carrington. 
    A and B mark the initial positions of an intensely bright event, 
    which moved over the course of five minutes to C and D before 
    disappearing. Source: Wikipedia}
    \label{fig:carringtonevent}
\end{wrapfigure}

The telegraph network in Europe and North America failed. Some operators experienced electric shocks 
\citep[pg.~13]{board2008committee} while in some cases even telegraph equipment that was disconnected 
from the power supply could be used to transmit messages \citep[pg.~58]{carlowicz2002storms}.

Based on global reports and observations taken by Scottish physicist Balfour Stewart at the 
Kew observatory in London, Carrington was able to connect events observed on the Earth to what 
he saw on the Sun on the $1^{\text{st}}$ of September \citep{clark2007sun}. His assertion was corraborated
by other observers in the scientific community.

The storm of 1859, later known as the \emph{Carrington event} was in some ways the genesis of \emph{Space Weather},
although the actual term was coined much later in the $1950$s. Although scientists had observed sunspots and their 
links to magnetic field variations on the Earth earlier, the \emph{Carrington event} was a concrete example of how 
activity on the Sun could have potentially dramatic effects on the Earth.

\subsection*{Space Weather}

How do spots and ejections from the Sun produce bright lights and currents on Earth?

\clearpage
\bibliographystyle{plainnat}
\bibliography{references}
