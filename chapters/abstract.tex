\chapter*{Summary}\label{chapter:abstract}

\begin{center}

    \textbf{\doctitle}

    \textit{\docsubtitle}


\end{center}

The study of variations in the space environment between the Sun and the Earth constitutes 
the core of space weather research. Ionized plasma ejected by the Sun couples with the Earth’s 
magnetic field in complex processes that determine the state of the Earth's magnetosphere. Adverse 
effects from space weather can impact communication networks, power grids and logistics 
infrastructure, all crucial pillars of a civilization that is reliant on technology.

It is important to use data sources, scientific knowledge and statistical techniques to create 
space weather forecasting and monitoring systems of the future. This thesis aims to be a step 
towards that goal. The work is organised into the following sections/chapters.

In \cref{chapter:dst_osa,chapter:dst_msa}, we develop probabilistic forecasting models for 
predicting geo-magnetic time series. Combining ground based and satellite measurements, we propose 
a gaussian process model for one hour ahead prediction of the Dst time series. We augment this 
model with a long short-term memory network and produce probabilistic predictions six hours ahead 
for Dst.

Quantifying uncertainties in the Earth's radiation belt parameters is an important step for 
producing ensembles of high fidelity simulations of the magnetosphere. In 
\cref{chapter:bayes_diff_chapter}, we combine simplified physical models of the radiation belt with 
Markov Chain Monte Carlo techniques to obtain uncertainties in magnetospheric parameters, using 
data from probes orbiting in the radiation belts.

In time-varying phenomena, it is often the case that effects of disturbances are not observed 
immediately, but after a certain time interval which can be dynamic. A prominent example this 
behavior is the Sun-Earth system. Particles ejected from the Sun, also called the solar wind, reach 
the Earth's magnetosphere after a time delay which is uncertain. We propose a novel neural network 
based method, called Dynamic Time Lag Regression (\XX), apply it to the problem of solar wind 
propagation.
