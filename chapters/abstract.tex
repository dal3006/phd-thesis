\chapter*{Abstract}\label{chapter:abstract}

The study of variations in the space environment between the Sun and the Earth constitutes 
the core of \textit{space weather} research. Ionized plasma ejected by the Sun couples with 
the Earth’s magnetic field in complex processes that determine the state of the Earth's 
magnetosphere. Adverse effects from space weather can impact communication networks, 
power grids and logistics infrastructure, all crucial pillars of a civilization that 
is reliant on technology.

It is thus important to leverage data sources, scientific knowledge and statistical learning 
methodology to create space weather forecasting and monitoring systems of the future. This 
thesis aims to be a step towards that goal. The work is organised into the following 
sections/chapters.

\begin{enumerate}
\item \textit{Forecasting}: We develop probabilistic forecasting models for predicting 
\textit{geo-magnetic} time series. Combining ground based and satellite measurements, 
we propose a \textit{gaussian process} model for one hour ahead prediction of the \textit{Dst} 
time series \cite{ChandorkarDst}, \cite{CHANDORKAR2018237}. We augment this model with a 
\textit{long short-term memory} network and produce probabilistic predictions six hours 
ahead for \textit{Dst} \cite{doi:10.1029/2018SW001898}.

\item \textit{Parameter Inference \& Uncertainty Quantification}: Quantifying uncertainties in the 
Earth's \textit{radiation belt} parameters is an important step for producing ensembles of high 
fidelity simulations of the \textit{magnetosphere}. We combine simplified dynamical models with 
Markov chain Monte Carlo techniques to infer uncertainties in magnetospheric parameters, 
using data from probes orbiting in the radiation belts.

\item \textit{Causal Time Lag Prediction}: In temporal phenomena, it is often the case 
that causal effects of events are not immediately observed, but after a certain time interval 
which can be dynamic. One prominent example of such behavior is the \textit{Sun-Earth} system. 
Particles ejected from the Sun, also called the \textit{solar wind}, reach the Earth's magnetosphere 
after a time delay which is uncertain. We propose a novel neural network based method, for 
predicting causal time delay between time series and apply it to the problem of 
\textit{solar wind} propagation.

\end{enumerate}

\bibliographystyle{plainnat}

\bibliography*{references}

