\chapter*{Summary}\label{chapter:abstract}

\begin{center}

    \textbf{\doctitle}

    \textit{\docsubtitle}


\end{center}

The study of variations in the space environment between the Sun and the Earth 
constitutes the core of space weather research. Plasma ejected by the 
Sun couples with the Earth’s magnetic field in complex ways that determine 
the state of the Earth's magnetosphere. Adverse effects from space weather can 
impact communication networks, power grids and logistics infrastructure, all 
crucial pillars of a civilization that is reliant on technology.

It is important to use data sources, scientific knowledge and statistical 
techniques to create space weather forecasting and monitoring systems of the 
future. This thesis aims to be a step towards that goal. The work is organised 
into the following chapters.

In \cref{chapter:dst_osa,chapter:dst_msa}, we develop probabilistic forecasting 
models for predicting geo-magnetic time series. Combining ground based and 
satellite measurements, we propose a gaussian process model for forecasting of 
the Dst time series one hour ahead. We augment this model with a 
long short-term memory (LSTM) network and produce six-hour-ahead probabilistic 
forecasts for Dst.

Quantifying uncertainties in the dynamics of the Earth's radiation belt is an 
important step for producing ensembles of high fidelity simulations of the 
magnetosphere. In \cref{chapter:bayes_diff_chapter}, we infer uncertainties in 
magnetospheric parameters, using data from probes orbiting in the radiation 
belts, by combining simplified physical models of the radiation belt with 
Markov Chain Monte Carlo techniques.

In time-varying systems, it is often the case that cause and effect don't occur
at the same time. A prominent example of this time-lagged behaviour is the 
Sun-Earth system. Particles ejected from the Sun, also called the solar wind, 
reach the Earth's magnetosphere after a time delay which is dynamic and 
uncertain. In \cref{chapter:pdt}, we propose a novel neural network based 
method, called Dynamic Time Lag Regression (\XX), for predicting time-lagged 
effects of events and apply it to the problem of near-Earth solar wind 
forecasting from heliospheric data.
