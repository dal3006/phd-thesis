\chapter{Outline}\label{chapter:Outline}

The study of variations in the space environment between the Sun and the Earth constitutes 
the core of \textit{space weather} research. Ionized plasma ejected by the Sun couples with 
the Earth’s magnetic field in complex processes that determine the state of the Earth's 
magnetosphere. Adverse effects from space weather can impact communication networks, 
power grids and logistics infrastructure, all crucial pillars of a civilization that 
is reliant on technology.

It is important to leverage data sources, scientific knowledge and machine learning 
techniques to create space weather forecasting and monitoring systems of the future. This 
thesis aims to be a step towards that goal. 

\section{Research Questions}

The research questions addressed in this dissertation center around three key themes.

\begin{enumerate}
    \item \textit{Forecasting}: How can we create probabilistic forecasting systems for key geomagnetic quantities? 
    What is the time horizon for such forecasts and how can we improve it, if possible?.
    
    \item \textit{Parameter Inference \& Uncertainty Quantification}: Are machine learning models and physics models two 
    separate pieces? Can they be a married together in a way which enables identification of unobserved physical parameters?  
    
    \item \textit{Probabilistic Time Lag Prediction}: How can we predict near Earth solar wind speed and its propagation time from solar data? 
    More generally, is it possible to infer using time series data, when the impacts of events will be felt at their downstream destinations? How 
    can we deal with the challenge that the propagation time is dynamic and uncertain?  
    
\end{enumerate}
    
\section{Chapter Outline}

The dissertation is organised as follows.

\begin{itemize}
    \item Chapter \ref{chapter:introduction} provides context for the space weather project and its 
    links to terrestrial weather via historical anecdotes. It highlights impacts of space weather events 
    and speculates why space weather research will become even more important in the coming decades.
    \item Chapter \ref{chapter:preliminaries} provides a short introduction to the concepts needed to 
    understand the problems considered in this thesis. Section \S~\ref{sec:mag} talks about the magnetosphere and 
    the motions of charged particles trapped within it, this is the starting point for the material in chapters 
    \ref{chapter:dst_osa}, \ref{chapter:dst_msa} \& \ref{chapter:bayes_diff_chapter}. Section \S~\ref{sec:solar} gives a 
    quick overview about the structure of the Sun, its magnetic field and the solar wind which is the target 
    application for the method proposed in chapter \ref{chapter:pdt}.
    \item Chapter \ref{chapter:dst_osa} applies \emph{gaussian process} (GP) models for making probabilistic forecasts of 
    Earth based geomagnetic quantities and gives a practical methodology for building and evaluating such models. 
    Chapter \ref{chapter:dst_msa} extends the time horizon of the forecasting models proposed in chapter \ref{chapter:dst_osa} 
    by using a hybrid model based on \emph{long short term memory} (LSTM) networks and gaussian processes.
    \item Chapter \ref{chapter:bayes_diff_chapter} proposes a model based on the \emph{least squares support vector machine} (LSSVM) 
    for estimating magnetospheric plasma density. The parameters of the model are estimated by minimizing error with respect to 
    observed data and the physical dynamics of plasma diffusion. Combining this with the \emph{markov chain monte carlo} procedure, we 
    obtain Bayesian estimates on unobserved parameters of the plasma diffusion system.   
    \item Chapter \ref{chapter:pdt} introduces \emph{dynamic time lag regression} (DTLR) a novel supervised regression framework 
    which captures probabilistic and dynamic propogation time delays between time series. We propose a solution methodology for 
    the DTLR setting and provide a theoretical framework for understading its convergence. The DTLR methodology is applied to the 
    forecasting of solar wind from solar magnetic data. 
    \item Chapter \ref{chapter:conclusions} discusses the progress made in the thesis and offers perspectives for future research.
\end{itemize}


Chapters \ref{chapter:dst_osa} \& \ref{chapter:dst_msa} are based on published journal articles while \ref{chapter:bayes_diff_chapter} \& 
\ref{chapter:pdt} are based on work which is in peer review or in preparation for publication.  