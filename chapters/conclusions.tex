\chapter{Concluding Remarks}\label{chapter:conclusions}

This thesis represents an exploration of the possibilities for using machine learning techniques 
for advancing space weather research. The work presented here was classified into three principal 
research problems or themes as mentioned in \cref{chapter:Outline}. Below we give a quick summary 
of the main achievements of this thesis and avenues for further research.

\section{Discussion}

\subsection*{Geomagnetic Time Series Forecasting}

Using Gaussian process auto-regressive methods, it is possible to obtain accurate and reliable 
probabilistic forecasts for the $\mathrm{Dst}$ index, up to five hours ahead. 

The GP-AR and GP-ARX models give a general framework for modeling non-linear dynamical systems and 
provide uncertainty estimates on their time evolution. Their main drawback is the $O(N^3)$ time 
complexity for performing inference which makes applications on large data sets challenging. By 
using neural network based models as mean functions of GP models, one can create hybrid models 
which somewhat circumvent this drawbacks while retaining the probabilistic forecasting 
capabilities.

\subsection*{Radiation Belt Parameter Inference}

Using machine learning models as surrogates for quantities governed by physical laws, one can 
obtain models with some desirable properties: the ability synthesise observations and prior 
knowledge of physical dynamics into a coherent methodology for parameter inference and uncertainty 
quantification. 

When performing inference over the parameters of the radiation belt dynamics, one needs to take 
into account the sensitivity of the radiation belt model to its parameters. It is advisable to use 
domain knowledge and sensitivity analysis to constrain the numerical ranges of the parameter 
prior distributions as this aides identification and obtaining compact uncertainty estimates.

Casting the surrogate optimisation problem (\cref{eq:surrogate}) in its dual form enables the use 
of potentially infinite dimensional basis function expansions but it introduces the same 
computational challenges that come with Gaussian process inference. 

\subsection*{Solar Wind Prediction}

The effect of time lag relationships between interacting systems can impact the performance of 
predictive models which give forecasts for a fixed time lag. We proposed a principled approach for 
training predictive models on time series data sets which have non-stationary time lag dependencies.

The task of predicting near-Earth solar wind speed from heliospheric data is very challenging and 
requires astute application of all the tools at our disposal: 
\begin{enumerate*} 
    \item models of the heliospheric magnetic field,
    \item machine learning techniques, and 
    \item solar and near-Earth data. 
\end{enumerate*}


\section{Further Research}

Although the space weather problem is the central motivation for the work presented here, the 
techniques presented in this thesis are generally applicable in the modeling and forecasting of 
physical systems. Some possible questions for further research are listed below.

\begin{enumerate}
    \item With the existing state of the art, how far we extend the time horizon of geomagnetic 
          activity forecasts? What kind of data and methods will be important in making 
          ten-hour-ahead or twelve-hour-ahead forecasts of the $\mathrm{Dst}$ index?
    \item How do we extend the phase space density surrogate model to higher dimensional 
          radiation belt dynamics i.e. diffusion across all three adiabatic invariants?
          How do we deal with the computational challenges that arise from performing inference 
          over parameters of $3$-d diffusion models and large scale data sets? 
    \item When working in the context of PDE constrained inverse problems, how do we separate 
          uncertainties arising from parameter identifiability and forward model inadequacy? How do 
          we perform inference over the parameters of a non-linear PDE using machine learning based 
          surrogate models?
    \item How can we improve the accuracy of solar wind forecasts made by the \XX \ model? How can 
          we make the combination of the CSSS and \XX \ models into a real time solar wind 
          forecasting system? Is it beneficial to use a surrogate in place of the CSSS model to 
          compute the topology of the HMF? 
\end{enumerate}

All things considered, there are plenty of directions for further research into applications of 
machine learning methods in space weather and the physical sciences.
