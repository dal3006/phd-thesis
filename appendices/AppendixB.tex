\chapter{Stability Analysis Of The \XX \ Model}\label{app:Hessian}
The log likelihood as a function of $r$ and $\beta=\alpha/r$ after inserting the optimal
$\p = \p(x,{\bf y})$ reads in that case
\[
{\cal L}(r,\beta) = -\frac{\vert T\vert}{2}\log(r)-\frac{\vert T\vert}{2r}+\frac{1}{2}\log(1+r\beta)+{\mathbb E}_{data}\Bigl[\log(Z)-\lambda(x)\sum_{i\in T} \hat p_i(x)\Bigr]
\]
with
\[
Z = \sum_i \hat p_i(x)\exp\Bigl(-\frac{\beta}{2\sigma_0^2}\Delta y_i^2(x)\Bigr),
\]
and where $\lambda(x)$ is a Lagrange multiplier which has been added to impose the normalization of $\hat {\bf p}$.
The gradient reads
\begin{align*}
  \frac{\partial {\cal L}}{\partial r} &= \frac{1}{2r^2}\Bigl(\vert T\vert(1-r)+\frac{\beta r^2}{1+\beta r}\Bigr),\\[0.2cm]
  \frac{\partial {\cal L}}{\partial \beta} &= \frac{r}{2(1+r\beta)}-\frac{1}{2}C_1[\p],\\[0.2cm]
  \frac{\partial{\cal L}}{\partial \hat y_i(x)} &= \frac{1}{\sigma^2}{\mathbb E}_{data}\Bigl[ \bigl(y_i-\hat y_i(x)\bigr)\bigl(1+\alpha q_i(x,{\bf y})\bigr)\Big\vert x\Bigr].\\[0.2cm]
  \frac{\partial {\cal L}}{\partial \hat p_i(x)} &= \frac{{\mathbb E}_{data}\bigl[ q_i(x,{\bf y})\vert x \bigr]}{\hat p_i(x)}-\lambda(x),
\end{align*}
with
\[
C_1[\p] = \frac{1}{\sigma_0^2}{\mathbb E}_{data}\Bigl(\sum_{i\in T} q_i(x,{\bf y})\Delta y_i^2(x)\Bigr),
\]
This leads to the following relation at the saddle point:
\begin{align*}
  r &= \frac{\vert T\vert-C_1[\p]}{\vert T\vert-1},\\[0.2cm]
  \alpha &= \frac{\vert T\vert}{\vert T\vert-1}\frac{1-C_1[\p]}{C_1[\p]},\\[0.2cm]
\hat y_i(x) &= \frac{{\mathbb E}_{data}\Bigl[y_i\bigl(1+\alpha q_i(x,{\bf y})\bigr)\Big\vert x\Bigr]}
{{\mathbb E}_{data}\Bigl[1+\alpha q_i(x,{\bf y})\Big\vert x\Bigr]}\\[0.2cm]
  \hat p_i(x) &= \mathbb{E}_{data}\bigl[ q_i(x,{\bf y})\vert x \bigr].
\end{align*}
Let us now compute the Hessian. It is easy to see that the block corresponding to the predictors $\hat{\bf y}$ decouples from the rest as soon as these predictors are centered. 

Denoting
\[
C_2[\p] = \frac{1}{\sigma_0^4}{\mathbb E}_{data}\Bigl[\sum_{i\in T} q_i(x,{\bf y})\Bigl(\Delta y_i^2(x)-\sum_{j=1}^n q_j(x,{\bf y})\Delta y_j^2(x)\Bigr)^2\Bigr],
\]
we have
\begin{align*}
\frac{\partial^2 {\cal L}}{\partial r^2} &= \frac{1}{2r^2}\Bigl(-{\vert T\vert} +2\frac{\vert T\vert}{\vert T\vert-1}\bigl(C_1[\p]-1\bigr)-\beta^2 C_1^2[\p]\Bigr)\\[0.2cm]
\frac{\partial^2 {\cal L}}{\partial r\partial \beta} &=\frac{1}{2r^2}C_1^2[\p]\\[0.2cm]
\frac{\partial^2 {\cal L}}{\partial \beta^2} &= \frac{1}{4}\Bigl(C_2[\p]-2C_1^2[\p]\Bigr)\\[0.2cm]
\frac{\partial^2 {\cal L}}{\partial \hat p_i(x)\partial\hat p_j(x)}  &= -\frac{{\mathbb E}_{data}\bigl[ q_i(x,{\bf y})q_j(x,{\bf y})\vert x \bigr]}{\hat p_i(x)\hat p_j(x)}\\[0.2cm]
\frac{\partial^2 {\cal L}}{\partial r\partial\hat p_i(x)} &= 0\\[0.2cm]
\frac{\partial^2 {\cal L}}{\partial \beta\partial\hat p_i(x)} &= -\frac{C_{2+i}[\p]}{2\hat p_i(x)},
\end{align*}
where
\[
C_{2+i}[\p] = \frac{1}{\sigma_0^2}{\mathbb E}_{data}
\Bigl[ q_i(x,{\bf y})\bigl(\Delta y_i^2(x)-\sum_{j\in T} q_j(x,{\bf y})\Delta y_j^2(x)\bigr)\vert x \Bigr].
\]
There are two blocks in this Hessian, the one corresponding to $r$ and $\beta$ and the one corresponding to derivatives with respect to $\hat p_i$.
The stability of the first one depends on the sign of $C_2[\p]-2C_1^2[\p]$ for $\vert T\vert$ large while the second block
is always stable as being an average of the exterior product of the vector $(q_1(x,{\bf y})/\hat p_1(x),\ldots,q_{\vert T\vert}(x,{\bf y})/\hat p_{\vert T\vert}(x))$
by itself. At the degenerate point $\alpha=0$, $r=1$, $\hat  p_i=1/\vert T\vert$ the Hessian simplifies as follows. Denote
\[
{d{\bf \eta}} = dr{\bf e}_1+d\beta{\bf e}_2+\sum_{i\in T} d\hat p_i(x){\bf e}_{i+2}
\]
a given vector of perturbations decomposed onto the canonical basis $\{{\bf e}_k,k=1,\ldots \vert T\vert+2\}$.
Denote
\begin{align*}
{\bf u} &= \sum_{i\in T} C_{2+i} {\bf e}_{2+i}\\[0.2cm]    
{\bf v} &= \sum_{i\in T}  {\bf e}_{2+i}
\end{align*}
with
\begin{align*}
C_2 &= \frac{1}{\vert T\vert\sigma_0^4}{\mathbb E}_{data}\Bigl[\sum_{i\in T} \Bigl(\Delta y_i^2(x)-\frac{1}{\vert T\vert}\sum_{j\in T}\Delta y_j^2(x)\Bigr)^2\Bigr].\\[0.2cm]
C_{2+i} &= \frac{1}{\sigma_0^2}{\mathbb E}_{data}
\bigl[\Delta y_i^2(x)-\sigma_0^2\vert x \bigr].
\end{align*}
With these notations the Hessian reads: 
\[
H = \frac{1}{2}\bigl(-\vert T\vert{\bf e}_1{\bf e}_1^t+{\bf e}_1{\bf e}_2^t+{\bf e}_2{\bf e}_1^t+\bigl(\frac{C_2}{2}-1\bigr){\bf e}_2{\bf e}_2^t
-{\bf u}{\bf e}_2^t-{\bf e}_2{\bf u}^t-{\bf v}{\bf v}^t\bigr).
\]
We see that $H$ is of rank four. In fact we can look at its eigenvalues in the subspace of deformations
which conserve the norm of $\hat {\bf p}$, i.e. orthogonal to ${\bf v}$, given by
\[
{\bf \eta} = \eta_1{\bf e}_1+\eta_2{\bf e}_2+\eta_3{\bf u}.
\]
In this subspace the Hessian reads
\[
H =\frac{1}{2}
\left[
  \begin{matrix}
 -\vert T\vert& 1 &0 \\[0.4cm] 
 1 & \qquad\DD \frac{C_2}{2}-1&\qquad-\vert {\bf u}\vert^2  \\[0.4cm]
 0& -\vert {\bf u}\vert^2 &0 
  \end{matrix}
  \right]
\]
If $|u|^2>0$ or  if $|u|=0$ and $1+\vert T\vert(C_2/2-1)>0$ there is at least one positive eigenvalue. Let $\lambda$ be such an eigenvalue.
The deformation along this mode reads
\[
{d\bf \eta} \propto \lambda{\bf e}_1+\lambda(n+\lambda){\bf e}_2-|u|^2(\vert T\vert+\lambda){\bf u},
\]
which corresponds to increasing $r$ and $\alpha$ while decreasing the $\hat p_i$ having the highest mean error $C_{2+i}$.


\noindent Concerning solutions for which
\[
\hat p_i(x) = \delta_{i\hat I(x)}
\]
is concentrated on some index $\hat I(x)$,  the analysis is more complex. In that case $C_2[{\bf p}]=0$ and $C_1[{\bf p}] >0$. The $(r,\beta)$ sector has $2$ negative eigenvalues,
while the $\hat {\bf p}$ block is $(-)$ a covariance matrix, so it has as well negative eigenvalues. The coupling between these  two blocks could however in principle generate in some cases some instabilities.

Still, the log likelihood of such solutions reads
\[
{\cal L} = -\frac{\vert T\vert}{2}\log(\sigma^2)+\frac{1}{2}\log(1+\alpha)-\frac{1}{2\sigma^2}{\mathbb E}_{data}\Bigl[\sum_{i\in T}\Delta y_i^2(x)\Bigr]-\frac{\alpha}{2\sigma^2}{\mathbb E}_{data}\Bigl[\Delta y_{I(x)}^2(x)\Bigr]
\]
so we get the following optimal solution
\begin{align*}
\sigma^2 &= \frac{1}{\vert T\vert}{\mathbb E}_{data}\Bigl[\sum_{i\in T}\Delta y_i^2(x)\Bigr],\\[0.2cm]
\frac{1}{1+\alpha} &=  \frac{{\mathbb E}_{data}\Bigl[\Delta y_{I(x)}^2(x)\Bigr]}{\sigma^2},\\[0.2cm]
I(x) &= \argmin_{i\in T} {\mathbb E}_{data}\Bigl[\Delta y_i^2(x)\vert x \Bigr].
\end{align*}